\documentclass{article}
\usepackage{amsmath}
\usepackage{amssymb}

\begin{document}

\textbf{Conjecture.} For any integer \( n \) that has 5 as the last digit, \( n^2 \)
will have 25 as the last two digits.

\medskip

\textbf{Proof.} Let \( n \) be an integer with 5 as the last digit.

Then \( n = 10j + 5 \) for some integer \( j \).

Therefore:
\begin{align*}
n^2 &= (10j + 5)(10j + 5) \\
    &= 100j^2 + 100j + 25 \\
    &= 100j(j + 1) + 25
\end{align*}

Let \( \ell = j(j + 1) \). Then we have:
\[
n^2 = 100\ell + 25
\]

which is to say:
\[
n^2 \equiv 25 \pmod{100}
\]

Since congruence mod 100 preserves the last two decimal digits, \( n^2 \) ends in 25.

\hfill $\square$

\end{document}
